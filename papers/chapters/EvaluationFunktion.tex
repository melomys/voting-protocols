\chapter{Evaluationmethode}

\section{Evaluationsfunktionen}
Sämtliche definierten Modellparameter können in der Auswertung einer agentenbasierten Modellierung betrachtet werden.
Es wird der \textit{Normalized Discounted Cumulative Gain}-Koeffizient (nDCG) verwendet, um die Güte einer Bewertungsmetrik unter den gegebenen Modellparametern zu berechnen. Außerdem wird der Gini-Koeffizient für jeden Modelllauf berechnet, dieser misst, wie fair die Sichtbarkeit von Posts verteilt ist. Durch den nDCG- und den Gini-Koeffizienten lassen sich unterschiedliche Modellkonfigurationen vergleichen.

\subsection{Discounted Cumulative Gain}

Der \textit{Discounted Cumulative Gain}-Koeffizient (DCG) in \cite{Biega2018405} dient zur Bewertung von Bewertungsmetriken. Der DCG in Formel \ref{dcg} bestraft Bewertungsmetriken, welche Posts mit hoher Qualität auf hintere Rankingpositionen einordnet. Die Qualität eines Posts fließt durch den Logarithmus der Rankingposition des Posts ins Verhältnis gesetzt in den DCG ein.

\begin{equation}
\label{dcg}
DCG(B) = \sum_{i = 1}^{N}\frac{2^{Q_{P,i}}-1}{log_2(i + 1)}
\end{equation} 

\paragraph{Normalized Discounted Cumulative Gain}

Der nDCG normalisiert den DCG einer Bewertungsmetrik mit einer idealen Bewertungsmetrik $B_I$, welche die Posts absteigend nach Qualität im Ranking anordnet. $IDCG = DCG(B_I)$ ist der ideale DCG. Mit diesem ergibt sich:

\begin{equation}
\label{ndcg}
nDCG(B) = \frac{DCG(B)}{IDCG} 
\end{equation}

Eine optimale Bewertungsmetrik erhält damit den Maximalwert $nDCG = 1$.

\subsection{Gini-Koeffizient}

In \cite{Lerman2014} und \cite{Salganik2006854} gibt der Gini-Koeffizient $G$ an, wie gerecht Aufmerksamkeit der User*innen auf die Posts verteilt ist.
Um die Aufmerksamkeit zu messen, wird die Anzahl der User*innen die einen Post betrachtet haben verwendet. Je größer der Gini-Koeffizient ist, desto ungerechter ist die Verteilung der Views. 
Bei $G = 0$ ist die Aufmerksamkeit gerecht verteilt. Alle Posts wurden von der gleichen Anzahl User*innen gesehen. Wenn ein Post von allen User*innen gesehen wurde, alle andere Posts hingegen von keine*r einzige*n User*in ist $G = 1$ maximal. Der Gini-Koeffizient wird beschrieben durch:

% TODO Da kommt nie 1 raus, sondern immer kleiner
\begin{equation}
G = \frac{1}{2S\sum_{i = 1}^{N}v_{i}}\sum_{i,j}|v_{i} - v_{j} |
\end{equation}





