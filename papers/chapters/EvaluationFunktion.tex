\chapter{Evaluationsfunktionen}


- ndg

[Equity of Attention paper]

- area under curve

- gini / gini under curve

Der Gini-Koeffizient gibt an, wie gerecht Aufmerksamkeit der User*innen auf die Posts verteilt ist. Dabei wird die Anzahl der User*innen die einen Post betrachtet haben verwendet. Je größer der Gini-Koeffizient ist, desto ungerechter ist die Verteilung der Views. Bei einem Gini-Koeffizienten von 1, wäre ein Post von allen User*innen betrachtet und alle anderen Posts von keinen User*innen gesehen worden. Bei einem Gini-Koeffizienten von 0 hingegen, wurden alle Posts von der gleichen Anzahl an User*innen gesehen.

\begin{equation}
G = \frac{1}{2S\sum_{i = 1}^{N}v_{i}}\sum_{i,j}|v_{i} - v_{j} |
\end{equation}

[Leveraging Position Bias to Improve Peer
Recommendation]

\cite{Lerman2014} \cite{Salganik2006854}


