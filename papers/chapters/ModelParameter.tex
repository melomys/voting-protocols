\chapter{Modellparameter}

\section{Qualität}

Qualität wird im Modell über eine kontinuierliche n-dimensionale Verteilung definiert. Als Modellierung bietet sich eine n-dimensionale Normalverteilung an. Einzelne Qualiätsmerkmale können so einfach in Korrelation gesetzt werden. Aus der Qualitätsverteilung wird sowohl die Qualität der Posts als die Qualitätswahrnehmung der User*innen gewonnen.


\section{Bewertungsverteilung}

Die Verteilung, die durch die Bewertungsfunktion von User*innen für Posts erzeugt wird ist nicht bekannt und kann nicht analytisch berechnet werden. Es ist jedoch notwendig Quantil der Bewertungsverteilung zu berechnen um zu entscheiden, ob ein*e User*in einen Post bewertet. Die Verteilung wird empirisch berechnet.

Aus der Qualitätsverteilung werden Pseudoposts mit Qualität und Pseudouser*innen mit Qualitätsperzeption erzeugt. Diese werden über ein kartesisches Produkt in die Bewertungsfunktion eingesetzt. Die so berechneten Bewertungswerte bilden die empirische Verteilung. 

[Vektorrechnung des kartesichen Produktes in die Bewertungsfunktion gestopft]

[Bild mit unterschiedlichen empirischen Bewertungsverteilungen, nach unterschiedlichen Ratingfunktionen]



\section{User*innenparameter}


\subsection{Qualitätsperzeption}

Die Qualitätsperpezeption beschreibt die Qualitätswahrnehmung im n-dimensionalen Qualitätsraum. Sie wird verwendet um mit der Bewertungsfunktion den Bewertungswert von User*innen für betrachtete Posts zu errechnen.

Die Qualitätsperzeption wird aus der Qualitätsverteilung gewonnen.

\subsection{Aktivtität}

In dem implementierten Modell beschreibt jede Iteration das Verstreichen eines Zeitschrittes (30 Minuten). Wie in \cite{Hogg20121} beschrieben sind nicht alle User*innen zu jedem Zeitpunkt aktiv. Die Aktivität von User*innen lässt sich durch eine logarithmische Normalverteilung
beschreiben. In diesem Modell wird die Aktivität über eine Wahrscheinlichkeit modelliert,  mit welcher ein*e User*in in einem Modellschritt aktiv ist. Da sich die logarithmische Normalverteilung nicht zufriedenstellend auf das Intervall $ [0,1]$ begrenzt werden kann, wird eine $\beta$-Verteilung zur Approximation verwendet.

[Bild mit unterschiedlichen $\beta$-Verteilungen]

\subsection{Bewertungszufriedenheit}


Über die Bewertungszufriedenheit wird festgelegt, wie (un)zufrieden ein*e User*in mit einem Post sein muss, um diesen zu bewerten. 
Die Bewertungszufriedenheit gibt das Quantil der Bewertungsverteilung an, welches von der Bewertungsfunktion der User*innen für Posts über- bzw. unterschritten werden muss um eine negative bzw. positive Bewertung der User*in für den Post hervorzurufen.

Wie die Aktivitätswahrscheinlichkeit wird Bewertungszufriedenheit von einer $\beta$-Verteilung modelliert.

\subsection{Konzentration}

Die Konzentration gibt an, wie viel Posts ein*e User*in betrachtet, falls sie in einem Modellschritt aktiv ist. User*innen betrachten in jedem Schritt die gleiche Anzahl an Posts.
Sie ist diskret-positiv über die User*innen verteilt. Eine mögliche Verteilung stellt die Poissonverteilung.

[Bild mit unterschiedlichen Poissonverteilungen??]
