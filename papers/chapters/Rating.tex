\chapter{User*innenrating}

In der Realität ist nicht bekannt wie User*innen Posts aufgrund ihrer eigenen Qualitätsperzeption und der Qualität des Posts wahrnehmen und nach welchen Kriterien User*innen entscheiden einen Post zu bewerten. Diese Ratingfunktion ist jedoch elementar um User*innenaktivität in einem sozialem Medium zu simulieren. 

User*innenratingfunktionen sollen folgende Kriterien erfüllen:

\begin{enumerate}
	\item $R(P,U) \in [0,1]$
	\item Bei $R(P,U) = 1$ wird der Post von der User*in als maximal gut empfunden
	\item Bei $R(P,U) = 0$ wird der Post von der User*in als maximal schlecht empfunden
\end{enumerate}

\section{Transformation der Qualitätsparameter}

Um Kriterium 1 zu gewährleisten wird eine Transformation der Qualitätsparameter vollzogen, welche diese auf das Intervall $[0,1]$ 
begrenzen. Außerdem ist es möglich das approximierte Ratingfunktionen positive Werte erfordern um zu funktionieren. Die Transformation wird durch die {Sigmoid}-Funktion in Formel \ref{sigmoid} vollzogen:

\begin{equation}
\label{sigmoid}
s(x) = \frac{1}{1 + e^{-\frac{1}{2}x}}
\end{equation}

Somit ergeben sich die transformierten Qualitäts- bwz Qualitätsperzeptionsvektoren der Post und User*innen durch die komponentenweise Anwendung von $s$ auf $q_P$ und $q_U$

\begin{align}
\tilde{q}_P = s(q_P) \\
\tilde{q}_U = s(q_U) 
\end{align}

\section{Approxamtion des User*innenratings}

Im folgenden werden zwei Ansätze eingeführt um die User*innenbewertungsfunktion $R(P,U)$ zu approximieren.
 
\subsection{Konsensrating}

Im Konsensrating wird davon ausgegangen, dass User*innen "gute" Posts eher als diese erkennen. Alle User*innen sind sich über die besonders guten und schlechten Posts einig und erkennen diese richtig. Bei durchschnittlichen Posts kann es auch zu Meinungsverschiedenheiten kommen, je nach der Qualitätsperzeption der User*innen.

Das Konsensrating ist eher auf einen Q\&A Platform oder einem Techforum wie Hacker News denkbar. Konstruktive Beiträge sind eher von destruktiven zu unterscheiden. User*innen die besonders gut in einem Thema sind haben an Beiträge zu diesem Thema möglicherweise höhere Ansprüche.


In Formel \ref{konsensexp2} ist ein Ratingfunktion, welche die genannten Punkte ausdrückt:

\begin{equation}
\label{konsensexp2}
R_K(P,U) = \frac{1}{N}\sum_{i = 1}^{N}\tilde{q}_{P,i}^{\tilde{q}_{U,i}}
\end{equation}



\subsection{Dissensrating}

User*innen empfinden Posts als "gut", die nach an ihrer eigenen Qualitätsperzeption liegen. Dadurch sind sich User*innen bei vielen Posts uneinig.

Das Dissensrating ist eher auf einer Diskussionsplattformen denkbar. Dort können die Meinungen auseinander gehen. User*innen haben nicht unbedingt das Interesse die Position der Gegenüber einzunehmen. 

Eine Funktion die die genannten Kriterien erfüllt ist die euklidische Distanz, daher ergibt sich in Formel \ref{dissens}:


%TODO Dissensrating validieren
\begin{equation}
\label{dissens}
R_D(P,U) = 1 - \sqrt{||(\tilde{q}_P - \tilde{q}_U)||^2}
\end{equation}
