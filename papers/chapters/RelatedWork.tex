\chapter{Verwandte Arbeiten}

MusicLab Experiement 

-Experimental Study of Inequality and Unpredictability in an Artificial Cultural Market


- Measuring and Optimizing Cultural Markets

Ein Experiment in dem Testpersonen unbekannte Lieder bewerten. Die Testpersonen werden in zwei Gruppen aufgeteilt. Eine \textbf{unabhängige} Gruppe und eine unter \textbf{sozialem Einfluss}. die \textbf{unabhängige} Grupee erhält keine weiteren Informationen, alle User kriegen eine zufällige Liste der Lieder angezeigt und können diese downloaden. Die Gruppe unter \textbf{sozialem Einfluss} wird zusätzlich in acht Welten unterteilt, jeder User kriegt eine Liste mit Liedern nach der Downloadzahl sortiert angezeigt. Die Downloadzahlen werden angezeigt. Die Autoren zeigen, dass unter sozialem Einfluss Ungleichheit und Unvorhersehbarkeit der Popularität von Objekten besteht.

Gruppenbezogene Fairness
Fairness ist Quotient aus qualität/sichtbarkeit


-Fairness and Transparency in Ranking (NICHT IN SCOPUS)

Stellt fest, dass Fairness zwischen Gruppen hergestellt ist, wenn der Quotient aus Sichtbarkeit und Nutzbarkeit gleich ist.

Definiert Kriterien für ein faires ranking

- Equity of Attention: Amortizing Individual Fairness in Rankings


Wirft den Blick von Gruppen- auf individuelle Fairness. Es wird ein Mechanismus vorgestellt, welcher die Relevanz von Beiträgen über eine Sequenz von Rankings berechnet. 

Hier wird ebenfalls ein ranking für rankings (DCG u. normalisiert) eingeführt.

- Fairness of Exposure in Rankings



- Measuring Fairness in Ranked Outputs


Betrachtet die Fairness von Rankings mit Einbeziehung der Verteilung von beschützten und nicht beschützten Kandidaten.
Die eingeführten Messfunktionen von Fairness werden genutzt um die Benachteiligung von nicht beschützten Kanditen systematisch auszuwerten.

- FA*IR: A Fair Top-k Ranking Alogrithm

Stellt eine Methode vor welche aus beschützten und nicht beschützten Kandidaten die besten k Kanditaten sucht. Für diese sind die nicht beschützten Kanditaten nicht benachteiligt.


- Ranking with Fairness Constraints

(lineare) Algorithmen zur Findung eines fairen Rankings

Hacker News und Reddit Scoring - Stoddard

-Popularity Dynamics in Intrinsic Quality in Reddit and Hacker News


Vergleich Aktivitätsranking

- Leveraging Position Bias to Improve Peer
Recommendation

Beschreibt ein Experiment, in dem Beiträge nach den unterschiedlichen Protokollen \textit{Zufällig},\textit{Popularität}, \textit{Akitvität} und \textit{Fixiert} gerankt werden. Die Verteilung der Bewertung wird mit dem Gini-Koeffizienten angegeben. im \textit{Fixed} und \textit{popularity} ranking herrscht die größte ungerechtigkeit.


- Randomize HN

Luu schlägt vor etwas Random Noise in das Ranking von Hacker News einzufügen, um die scharfe Kante zwischen Posts die auf der Startseite landen und diesen, die keine Aufsmerksamkeit erhalten auszuglätten.

Few-get-richer-effekt

- The few-get-richer: a surprising consequence ofpopularity-based rankings


Der few-get-richer Effekt beschreibt das Phänomen, dass in einer kleinen Objektklasse, jedes einzelne Objekt mehr Aufmerksamkeit erhält und somit "reicher" werden


Reddit Kommentare Best Ranking

-How Not To Sort By Average Rating (Internet)

Der Reddit "Best" - Kommentar sortierung basiert auf der unteren Grenze des Wilson-Intervalls zur Berechnung des Konfidenzintervalls der Binomialverteilung. Diese wird hier vorgestellt.


-Beyond Accuray: Data Quality

Beschreibt die Dimensionen von Datenqualität die für die Nutzer*innen wichtig sind. Die 20  gefundenden Qualitätsdimensionen werden in 4 Kategorien in einem Framework zusammengefasst.


- Towards Quality Discourse in Online News Comments

Untersucht Kommentarsysteme von Nachrichtenagenturen und die Wirkung von Beiträgen mit niedriger Qualität auf Nutzer*innen und Journalist*innen. Außerdem wird gezeigt, wie individuelle Lesemotivation Einfluss auf die Qualitätswahrnehmung haben. Um die Qualität zu verbessern werden unter anderem Markierungs- und Moderationsstrategien vorgeschlagen.

- Understanding and Classifying the Quality of Technical Forum Questions

Schlägt eine Methode vor um die Qualität von Beiträgen in Q&A Foren automatisiert zu berechnen. Als Datensatz wird ein Stack Overflow Dump verwendet


- How to Count Thumb-Ups and Thumb-Downs:User-Rating based Ranking of Itemsfrom an Axiomatic Perspective

Für Rankings mit den Bewertungsmöglichkeiten "Daumen hoch" und "Daumen runter" stellt [] intuitive Axiome vor, welche eine sinnvolle Scoringfunktion  erfüllen sollte. Bekannte Scoringmethoden werden auf die Axiome geprüft und eine Methode vorgestellt, welche die Axiome erfüllt.

- Distributed Moderation Systems: An Exploration of Their Utility and the Social Implications of Their Widespread Adoption

Mill untersucht die Aktivität von Usern auf Reddit und stellt Zusammenhänge mit dem Votingverhalten fest.

(Diese Mega ausführliche Doktorarbeit)

- How Public Opinion Forms

Das Anzeigen Popularität von Beiträgen hat Einfluss auf das Bewertungsverhalten, so sammeln gute bzw. populäre Beiträge mit steigender Aufmerksamkeit auch zunehmend schlechte Bewertungen.

- Social Influence Bias:A Randomized Experiment Häufig zitiert!

Sagt, dass sozialer Einfluss die Bewertungsdynamiken verzerrt. Negativer sozialer kann von durch die Gruppenintelligenz ausgeglichen werden


- Popularity Dynamics and Intrinsic
Quality in Reddit and Hacker News


Definiert Qualität als die Anzahl von Votes, die ein Artikel erhalten würde, wenn die Artikel in zufälliger Reihenfolge ohne sozialen Einfluss, wie der aktuelle Score angezeigt würden. Sie finden heraus, dass in Reddit und Hacker News die Qualität von Posts und die Anzahl an Votes korreliert sind, Mit einem Spearman-Korrelationskoeffizienten von 0.8 bei Hacker News und einem Koeffizienten zwischen 0.54 und 0.75 für unterschiedliche Subreddits

- Social dynamics of Digg / Using a model of social dynamics to predict popularity of news

Bauen ein stochastisches Nutzer*innen model um die Popularität von Beiträägen in Digg vorherzusagen. Modelparameter, wie die Aktivitätsfunktion von Nutzer*innen werden aus einem Datensatz von Digg abgeleitet. Das Modell wird validiert und erfasst die Hauptkomponenten der Bewertungsdynamiken in Digg.

- Description and Prediction of Slashdot Activity


OPINION DYNAMICS:


Untersucht, meist agentenbasiert, wie die unterschiedlichen Agenten sich auf einen Konsens einigen können.


-Agent-Based Models for Opinion Formation: A Bibliographic Survey

Bibliographische Analyse zur agentenbasierten Modellierung von Meinungsdynamiken in Netzwerken.

- Opinion Dynamics of Skeptical Agents

Präsentieren ein agentenbasiertes Modell, welches skeptizistische Agenten gegenüber anderen Meinungen beinhaltet. Es wird gezeigt, dass es auch in solchen Konstellationen ein Konsens gefunden werden kann.