\chapter{Bewertungsmetriken}
\label{chapter:bewertungsmetriken}

Anhand von Bewertungsmetriken sollen Posts nach ihrer Relevanz sortiert werden. Ein Post $P$ erhält durch eine Bewertungsmetrik den Score $s_P = B(P,i_M)$ zum Zeitpunkt $i_M$ zugewiesen. Mit diesem zugewiesenen Score werden die Posts sortiert und auf der Plattform zur Verfügung gestellt.


Die nicht beobachtbaren Qualitätsparameter stehen in der Berechnung der Bewertungsmetrik nicht zur Verfügung, diese kann nur die beobachtbaren Parameter verwenden:


\begin{itemize}
	\item $a_P$, Veröffentlichungszeitpunkt
	\item $b_P$, Bewertungsvektor
	\item $s_P$, Score
	\item $v_P$, Anzahl der Betrachtungen
\end{itemize}

In einer idealen Bewertungsmetrik $B$ werden die Matthäus-Effekte reduziert, sie ist fair, wenn die folgenden Punkte erfüllt sind:

\begin{itemize}
	\item für zwei Posts $P_1$ und $P_2$ mit $r_{P_1} > r_{P_2}$ ist $B(P_1,i_M) > B(P_2,i_M)$, zum Zeitpunkt $i_M$
	\item Für jeden Post $P_i$ gilt $v_{P_i} > 1$, er wurde von mindestens eine*r User*in wahrgenommen
\end{itemize}

\section{Bewertungstransformation}
\label{voteevaluation}

Die Bewertungsvektoren liegen im $V_N$-dimensionalen Raum vor. Innerhalb von Bewertungsmetriken werden die $V_N$ dimensionalen Bewertungsvektoren von Posts auf einen Skalar transformiert. Dies geschieht mit einer Bewertungstransformationsfunktion $\tau: \mathbb{R}^{V_N} \rightarrow \mathbb{R}$.

Für den zweidimensionalen Fall mit Up- und Downvotes (${n_\uparrow}_P$, ${n_\downarrow}_P$) werden im folgenden einige Bewertungstransformationen vorgestellt.

\subsection{Differenz}

Es wird die Differenz aus Up- und Downvotes zu berechnet, so ergibt sich in Formel \ref{diff}:

\begin{equation}
\label{diff}
\tau_{diff}(b_P) = {n_\uparrow}_P - {n_\downarrow}_P
\end{equation}

Die Differenz wird in der Reddit Hot Metrik standardmäßig verwendet.

\subsection{Anteil}

Berechnet wird der Anteil der Upvotes zur Gesamtzahl der Votes:

\begin{equation}
\label{anteil}
\tau_{anteil}(b_P) = \frac{{n_\uparrow}_P}{{n_\uparrow}_P + {n_\downarrow}_P}
\end{equation}


\subsection{Wilson Score}

Wie in \cite{miller} erläutert wird, ist die Verwendung der beiden vorherig vorgestellten Bewertungstransformationen in bestimmten Konstellationen von Up- und Downvotes nicht sinnvoll. Es wird  die untere Grenze des Wilson-Konfidenzintervall für die Erfolgswahrscheinlichkeit $p$ der Binomialverteilung $p$ vorgeschlagen.% Dabei ist $n$ die Gesamtanzahl an abgegeben Votes an einen Post, die Stichprobengröße, und $k$ die Anzahl an positiver Bewertungen eines Postes, die Anzahl an Erfolgen in der Stichprobe.
	
Mit der Gesamtzahl der Bewertungen eines Posts $n_P = {n_\uparrow}_P + {n_\downarrow}_P$, dem Punktschätzer $\hat{p}_P = \tau_{anteil}(b_P)$ und dem Quantil $c_{\alpha}$ der Normalverteilung zum Irrtumsniveau $\alpha$ ergibt sich in Formel \ref{wilson} der Wilson Score:

\begin{equation}
\label{wilson}
 \tau_{wilson}(b_P) = \frac{1}{1+\frac{c_{\alpha}^2}{n_P}}(\hat{p}_P + \frac{c_{\alpha}^2}{2n_P} - c \sqrt{\frac{\hat{p}_P(1 - \hat{p}_P)}{n_P} + \frac{c_{\alpha}^2}{4n_P^2}})
\end{equation}
	
\section{Bewertungsmetriken}

Bewertungsmetriken berechnen den Score $B(P,i_M)$ für einen Post unter Betrachtung der Postparameter eines Postes $P$ zum Zeitpunkt bzw. Modellschritt $i_M$.
	
\subsection{Hacker News Metrik}
\label{seqHackerNews}

In \cite{SalihefendicHN} wird die Hacker News Metrik beschrieben. Die Upvotes ${n_\uparrow}_{P}$ eines Postes werden ins Verhältnis zum Alter $a_{P} = i_M - t_P$ gesetzt. Das Alter wird mit der Gravitätskonstante $\gamma = 1.8$ potenziert. Die Bewertungstransformation lautet $\tau(b_P) = {n_{\uparrow}}_P - 1$, so ergibt sich die Hacker News Bewertungsmetrik in Formel \ref{HackerNews}:
 
\begin{equation}
\label{HackerNews}
B_{HN}(P,i_M) = \frac{{n_\uparrow}_P - 1}{(a_P + 2)^{\gamma}}
\end{equation}

\subsection{Verallgemeinerte Hacker News Metrik}
\label{seqvHackerNews}
Das verallgemeinerte Hacker News Ranking verwendet eine beliebige Bewertungstransformation $v$ und lässt sich somit auch auf höhere Bewertungsräume anwenden:

\begin{equation}
\label{vHackerNews}
B_{vHN}(P,i_M) = \frac{\tau(b_P)}{(a_P + 2)^{\gamma}}
\end{equation}


\subsection{Reddit Hot Metrik}

Die Reddit Hot Metrik wird in \cite{SalihefendicR} beschrieben, hier wird ebenfalls die transformierte Bewertung $\tau{b_P}$ eines Posts mit dem Alter verrechnet.

Der Zeitwert $e_P$ wird als Differenz in Sekunden von diesem Zeitpunkt $i_M$ zum Zeitpunkt $E$ am 8.12.2005 um 07:46:43 in Formel \ref{RedditHotT} berechnet.

\begin{equation}
\label{RedditHotT}
e_{P} = i_{M} - E  
\end{equation}

Die Bewertungstransformation de Reddit Hot Metrik ist standardmäßig die  Differenz aus Up- und Downvotes, kann jedoch variiert werden:

\begin{equation}
\label{RedditHotD}
\tau(b_{P}) = \tau_{diff}(b_{P})
\end{equation}
\\
Der Parameter $z$ wird in \ref{RedditHotZ} auf das Maximum des Betrages von $ \tau(b_{P})$ aus Formel \ref{RedditHotD} und 1 gesetzt.

\begin{equation}
\label{RedditHotZ}
z_{P}  = \begin{cases}
|\tau(b_{P})| &\text{falls $|\tau(b_{P})| \geq 1$}\\
1 &\text{sonst}
\end{cases}
\end{equation}

Es ergibt sich die Reddit Hot Metrik in Formel \ref{RedditHot}:

\begin{equation}
\label{RedditHot}
B_R(P,i_M) = sign(\tau(b_{P})) * log_{10}(z_{P}) + \frac{e_{P}}{4500}
\end{equation}

Somit werden Posts mit fortschreitender Zeit nicht schlechter bewertet, wie bei der Bewertungsmetrik von Hacker News. Neuere Posts erhalten durch das Ansteigen von $e_P$ eine höhere Bewertung, bei gleichem $\tau(b_P)$, als ältere Posts. 
 

\subsection{Viewmetrik}

Die Viewmetrik wurde in dieser Arbeit entwickelt und basiert auf der in Abschnitt \ref{seqvHackerNews} beschriebenen Metrik. Es fließt die Anzahl der Betrachtungen des Posts $v_P$ mit ein. Je weniger Betrachtungen für ein gleiches $\tau(b_P)$ benötigt sind, desto besser bewertet die Viewmetrik. Daraus ergibt sich mit dem gleichen $a_P$ wie in Abschnitt \ref{seqHackerNews}  die Bewertungsmetrik in Formel \ref{ViewScoring}:

\begin{equation}
\label{ViewScoring}
B_V(P,i_M) = \frac{\frac{\tau_{P}(b_P)}{v_{P} + 1}}{(a_{P} + 2)^{\gamma}}
\end{equation}

\subsection{Aktivitätsmetrik}
%TODO genauer beschreiben
Diese Metrik basiert ebenfalls auf der verallgemeinerten Hacker News Metrik in Abschnitt \ref{seqvHackerNews} und wurde in dieser Arbeit entwickelt.
Die Bewertungstransformation des Posts wird mit dem Score der letzten Iteration verrechnet und durch das Alter skaliert. Die Aktivitätsmetrik wird in Formel \ref{Activation}

\begin{equation}
\label{Activation}
B_A(P, i_M) = \frac{\tau(b_{P}) - B_A(P,i_M-1)}{(a_{P} + 2)^{\gamma}}
\end{equation}


\section{Zufallsbewertung}

Nachdem Posts durch die Bewertungsmetrik bewertet wurden, kann der Score durch Zufall verunreinigt werden um den in \cite{Luu} vorgeschlagenen Lärm zur Bewertung hinzuzufügen. In dieser Arbeit wurden zwei unterschiedliche Ansätze zur zufälligen Verunreinigung gewählt.

\subsection{$\mu$-Abweichung}

Sei $\mu_s$ der Mittelwert der Scores aller Posts. Für einen Post $P$ mit Score $s_P$ wird die Verunreinigung $d$ als Zufallszahl im Intervall $d_{\mu,P} \in [-|\mu_s - s_P|,|\mu_s - s_P|]$ definiert. So ergibt sich für den verunreinigten Score $\tilde{s}_P$ des Post in Formel \ref{mean_deviation}:

\begin{equation}
\label{mean_deviation}
\tilde{s}_{\mu,P} =  s_P + d_{\mu,P}
\end{equation} 

\subsection{$\sigma$-Abweichung}

Für die Standardabweichung der Scores aller Posts $\sigma_s$ sei die Verunreinigung eine Zufallszahl im Intervall $d_{\sigma,P} \in [-\sigma_s,\sigma_s]$, sodass sich der verunreinigte Score eines Postes in Formel \ref{std_deviation} ergibt:

\begin{equation}
\label{std_deviation}
\tilde{s}_{\sigma,P} = s_P + d_{\sigma,P}
\end{equation}

	