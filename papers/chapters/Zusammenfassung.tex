\markboth{}{}
\paragraph{Zusammenfassung}
Die riesige Anzahl an Beiträgen auf sozialen Medien ist für Menschen nicht überschaubar. Um Nutzer*innen die interessanten und relevanten Beiträge anzuzeigen müssen diese vorsortiert werden. Für die Sortierung werden Bewertungsmetriken verwendet, diese sortieren Posts anhand ihrer Eigenschaften, zum Beispiel der Anzahl und Art der Bewertungen, die sie von den Nutzer*innen erhalten haben.

In dieser Arbeit wird ein agent*innenbasiertes Modell von Votingsystemen in der Onlinekommunikation entwickelt, mithilfe dessen unterschiedliche Bewertungsmetriken nach einem definierten Fairnessbegriff verglichen werden. Dazu werden einige Bewertungsmetriken beschrieben und auf unterschiedlichen Arten von Plattformen, wie News- und Q\&A-Plattformen, in der Simulation einer Onlinekommunikationsplattform angewendet und ausgewertet. Die getesteten Bewertungsmetriken unterscheiden sich in ihrer Fairness. Die Wahl der Bewertungsmetrik hat somit einen relevanten Einfluss auf das soziale Medium.
	
	