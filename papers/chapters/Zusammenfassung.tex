	\paragraph{Zusammenfassung}
	Die Anzahl an Beiträgen auf sozialen Medien ist für keinen Menschen überschaubar. Um ein gutes Benutzungserlebnis zu gewährleisten werden die Posts vorsortiert, dabei sollen gute und interessante Posts vielen User*innen angezeigt werden. Für die Sortierung werden Bewertungsmetriken verwendet, welche die Posts anhand ihrer Eigenschaften, zum Beispiel der Anzahl und Art der Bewertungen, sortieren. In dieser Arbeit wird ein Framework zur agent*innenbasiertes Modellierung einer Social Media Plattform entwickelt. Dabei werden sowohl Diskussionsplattformen als auch Wikis und Newsseiten betrachtet. Auf diesen wird das unterschiedlicher Bewertungsmetriken simuliert und verglichen. Dazu wird die Fairness einer Bewertungsmetrik definiert und schließlich ausgewertet. Es zeigt sich, dass die untersuchten Bewertungsmetriken unterschiedlich fair sind und die Wahl der idealen Bewertungsmetrik abhängig von der Art der Plattform ist.
	
	