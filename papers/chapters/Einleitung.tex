\chapter{Einleitung}

\section{Motivation}

Im letzten Jahrzehnt sind soziale Medien zu einem wichtigen Teil in der Kommunikation herangewachsen. Im Gegensatz zu konventionellen Medien können Nutzer*innen von sozialen Medien selbst Inhalte veröffentlichen, dadurch gibt es es viel mehr Beitragende als je zuvor. Dies führt zu einer Menge an Information, die kein Mensch überblicken oder verstehen kann. Von Social Media Plattformen werden Algorithmen verwendet, welche die Beiträge filtern und den Nutzer*innen diejenigen anzeigen, die sie wahrscheinlich interessant finden \cite{Sarwar2001285}. 

Um Informationen darüber zu generieren, welche Beiträge für Nuzter*innen relevant sind, verwenden viele soziale Medien und Onlineplattformen \textit{Votingsysteme}, um Nutzer*innen die Möglichkeit zu geben ihre Meinung über einzelne Beiträge zu äußern. Hacker News\footnote{\texttt{https://news.ycombinator.com}} erlaubt Nutzer*innen Posts nur positiv, mit Upvotes, zu bewerten. Auf Reddit\footnote{\texttt{https://www.reddit.com}} ist es ebenfalls möglich Posts mit Downvotes herabzuwerten. Onlineshops wie TheCubicle\footnote{\texttt{https://www.thecubicle.com}} verwenden zum Beispiel ein 1 bis 5 Sterne System um Artikel bewerten zu lassen.

Um Nutzer*innen Beiträge personalisiert anzuzeigen können Empfehlungssysteme verwendet werden. Diese zeigen jene Beiträge an, welche andere Nutzer*innen mit ähnlichen Interessen bereits als gut bewertet haben. Während Nutzer*innen aus ihrer Perspektive interessante Posts erhalten, hat diese Methode einige negative Effekte. Die Anwendung, so die Theorie, führt zur Bildung von Filterblasen \cite{Flaxman2016298}. Nutzer*innen interagieren nur noch mit Beiträgen, die sie positiv wahrnehmen. Nutzer*innen, die sich zum Beispiel in der Filterblase eines politischen Extrems befinden, erhalten so keine kritischen Beiträge mehr zu ihrer eigenen Meinung, wodurch sich diese verstärkt. 

Filterblasen können vermieden werden, indem allen Nutzer*innen die gleichen Beiträge angezeigt werden. Dennoch müssen die Beiträge anhand des Interesses für Nutzer*innen gefiltert und sortiert werden. Dazu werden \textit{Bewertungsmetriken} verwendet, die die Posts anhand ihrer Eigenschaften, zum Beispiel unter Einbeziehung ihrer Upvotes und des Veröffentlichungszeitpunkt, bewerten und anordnen.

Eine Problematik, die allerdings auch bei Bewertungsmetriken auftritt, ist der \textit{Matthäus-Effekt}: Beiträge, die bereits viel Aufmerksamkeit erhalten haben, werden deshalb weiteren Nutzer*innen angezeigt und erhalten dadurch noch mehr Aufmerksamkeit und Interaktion. Beiträge hingegen, die wenig Aufmerksamkeit erhalten, werden durch die Bewertungsmetrik oder das Empfehlungssystem nicht weiter als interessant erachtet und keinen weiteren Nutzer*innen angezeigt \cite{Stoddard2015416}.

Im Gegensatz zu Empfehlungssystemen, basieren Bewertungsmetriken meist nicht auf Methoden des maschinellen Lernens, sondern auf einfachen Algorithmen, die leicht austauschbar sind \cite{Adomavicius2005734}.

Es ist wünschenswert eine Bewertungsmetrik zu finden, die die Matthäus-Effekte minimiert. Aufgrund ihrer einfachen Struktur können sie leicht simuliert und verglichen werden. Es ist nicht ausgeschlossen, dass das Votingssystem Einfluss auf die Matthäus-Effekte hat. Das Votingsystem sollte auf der Suche nach einer optimalen Bewertungsmetrik mit einbezogen werden.

In dieser Arbeit wird ein agent*innen basiertes Modell von Votingsystemen in sozialen Medien entworfen, um unterschiedliche Bewertungsmetriken zu vergleichen.

\section{Agent*innenbasierte Modellierung}

In der agent*innenbasierten Modellierung werden die Aktionen und Interaktionen einzelner Entitäten, der Agent*innen, simuliert. Auch die Umwelt, in der die Agent*innen interagieren, wird modelliert. Mit Agent*innen können unterschiedliche Dinge beschrieben werden. Sie sind nicht auf die Modellierung von Menschen beschränkt. Durch Agent*innen können zum Beispiel in einer Simulation eines Vogelschwarms die Vögel modelliert werden. Für die Modellierung eines Waldbrandes würden die Agent*innen Bäume repräsentieren. Einzelne Agent*innen können dabei in ihren Eigenschaften sehr unterschiedlich sein. Werden Menschen durch die Agent*innen modelliert, können diese zum Beispiel extrovertiert oder introvertiert sein, eigennützig oder im Sinn der Gruppe handeln.


Wie in \cite{Burbach20203} beschrieben sind agent*innenbasierte Modelle weder absolut realistisch noch vollständig. Durch sie wird eine vereinfachte Realität modelliert. Die Agent*innen können zwar beliebig komplex parametrisiert werden, dennoch werden meist sehr einfache Verhaltensregeln der Agent*innen definiert. Denn durch die Interaktion entstehen bereits hier komplexe Systeme, welche die Realität ausreichend approximieren.

Kleine Veränderungen am Verhalten der Agent*innen können bereits große Veränderungen des Simulationsergebnisses hervorrufen. In der Simulation treffen die Agent*innen individuell unterschiedliche Entscheidungen, die durch die Verhaltensregeln vorgegeben sind und durch die Wahrnehmung der Agent*innen der Umwelt beeinflusst wird.

Die Agent*innen können in einem Netzwerk angeordnet sein, sodass sie sich gegenseitig in ihrem Verhalten beeinflussen, sie können jedoch auch nur mit dem System, in dem sie sich befinden, interagieren.
%Wie [Agents.jl] beschreibt können viele komplexe Systeme nicht vollständig durch herkömmliche mathematische Methoden beschrieben werden. Komplexe Systeme hängen von dem Verhalten der Elemente, der Agenten, des Systems ab. Kleine Änderungen am Verhalten der Agenten können große Veränderungen des Gesamtsystems hervorrufen. Durch agentenbasierte Modellierung können nicht lineare Modelle beschrieben.

%Agenten können in einem Raum angeordnet werden, sodass sie mit ihren Nachbarn interagieren, wie im Modell von [Shelling], oder als Netzwerk indem jeder Agent mit jedem Agenten interagieren kann. Möglich ist auch, dass die Agenten nicht untereinander, nur mit dem System, interagieren.

Die Popularität von agent*innenbasierten Modellen steigt stetig, sie sind beliebt, um unter anderem biologische, ökonomische und soziale Systeme, wie zum Beispiel soziale Medien zu modellieren. Auch in dieser Arbeit wird die agent*inenbasierten Modellierung zur Simulation einer Social Media Plattform verwendet.