\chapter{Ausblick}

- Sachen die noch berüchsigt werden könnten

Kommentare auf Posts

Im entwickelten Modell nimmt nur die Position im Rating sozialen Einfluss auf die Bewertungsentscheidung von Usern. Es wäre außerdem interessant den Einfluss zu untersuchen, welcher durch die Anzeige der aktuellen Bewertung des Posts entsteht

Dies wäre zu bewerkstelligen durch eine komplexere Kalkualtion der Bewertungswahrscheinlichkeit, in die dann nicht nur der skalare Wert des Users einfließt, sondern ebenfalls der betrachtete Post mit einfließt.

Dirichlet Smoothing



Konzentration besser modellieren und mit anderen User parametern korrelieren

Die Qualitätswahrnehmung von Usern ist im Modell nicht mit der Akitvität und der Bewertungswahrscheinlichkeit korreliert. Dies könnte jedoch in der Realität durchaus der Fall ist. Daher ist es vorstelllbar in einer Weiterentwicklung des Modells die Verteilungen zu korrelieren. 
Korrelationen von useraktivität, qauli

Eine einfache Umsetzung wäre denkbar, wenn die Dimensionsanzahl durch das Modell festgelegt ist